Economic change generates labor market frictions that vary across occupations and workers, yet most models abstract from how job search behavior adapts to uncertainty over the unemployment spell. This paper studies how adaptive search interacts with occupational structures and aggregate conditions to shape labor market adjustment. We extend a leading data-driven network model of occupational mobility by embedding empirically grounded dynamics in search effort, reservation wages, belief updating, and competition-reactive on-the-job search, grounded on U.S. micro data. Incorporating adaptive behavior improves the model's ability to replicate persistent long-term unemployment, vacancy–unemployment decoupling in recoveries, and heterogeneous wage outcomes following displacement. In doing so, we shed light on the role of behavioral adaptation in shaping labor market persistence and inequality.
%Whether driven by recessions, industrial policy, or structural shifts in labor demand, economic change is inevitably accompanied by labor market frictions. These frictions vary across occupations, demographic groups, and regions, increasing the risk of certain populations being left behind in times of economic transformation. 
%Such challenges to labor market stability are often studied as exogenous shocks taking the preferences and actions of workers to be fixed or negligible. However, incorporating the adaptive behavior taken by workers in the labor market to cope with uncertainty is critical to understanding both aggregate labor market fluctuations and adverse labor market outcomes related to wage and unemployment duration. 

%This study advances the capabilities of a leading data-driven network model of worker transitions by integrating behavioral foundations (dynamic search effort, duration-dependent wage expectations, subjective belief updating, and competition-reactive on-the-job search) grounded in US micro data. By incorporating a more comprehensive representation of worker decision-making, we enhance the model’s ability to simulate the heterogeneous social and economic impacts of economic change and disequilibrium. In doing so, it enables a deeper exploration of critical labor market phenomena, including long-term unemployment dynamics, gender wage disparities, and the uneven distribution of wage gains during periods of economic recovery. 