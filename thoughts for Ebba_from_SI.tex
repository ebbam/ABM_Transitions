\documentclass{article}
\usepackage{amsmath,amssymb}
%\usepackage{xcolor}


\begin{document}


\section*{A Model of Application Decisions with Risk Aversion 2}

We consider two populations of job seekers:
\[
  \mathcal{E} = \{\text{employed agents}\}, 
  \quad
  \mathcal{U} = \{\text{unemployed agents}\}.
\]

There are \(V\) vacancies and \(U\) unemployed, define labour?market tightness as
\[
  \varphi \;=\;\frac{V}{U}.
\]

\subsection*{1. Utility Functions}
Each agent \(i\) has a von Neumann Morgenstern utility over wages \(w\):
\[
  u_i(w) \;=\;
    \begin{cases}
      \dfrac{w^{1-\gamma_i}}{1-\gamma_i}, & \gamma_i \neq 1,\\[1em]
      \ln w, & \gamma_i = 1,
    \end{cases}
\]
where
\(\gamma_i>0\)  i.e.  risk averse (concave),  
\(\gamma_i<0\) i..e  risk seeking (convex),  
\(\gamma_i=0\) i.e risk neutral (\(u_i(w)=w\)).

\subsection*{2. Reservation Wage and Utility}
Each agent \(i\) faces a \emph{reservation wage} \(R_{i,t}\).  We posit
\begin{align}
  R_{E}(\varphi) 
    &= w_i \;\bigl[\,1 + \alpha_E\,(1 - \varphi)\bigr],
    \quad
    i \in \mathcal E,
    \label{eq:R_E}\\
  R_{U,t}(\varphi, D_{U,t})
    &= \underbrace{\,\overline w}_{\text{base level}}
      \;\bigl[\,1 - \alpha_U\,(1 - \varphi)\bigr]
      + \delta_U\,D_{U,t},
    \quad
    i \in \mathcal U.
    \label{eq:R_U}
\end{align}
- In a recession (\(\varphi\!\downarrow\)), \(R_{E}\) \emph{rises} (employed hold on),
  while \(R_{U}\) \emph{falls} (unemployed apply even to low wages).
- \(D_{U,t}\) is an \emph{unemployment discouragement index} (see below), and
  \(\delta_U>0\) captures how failure to find work raises the unemployed's  bar.
Their \emph{reservation utility} is then \(u_i\bigl(R_{i,t}\bigr)\).




\subsection*{3. Expected Utility of Applying}
Suppose any vacancy \(j\) yields a match with probability 
\(p(\varphi)\) (decreasing in tightness).  If matched, agent \(i\)
earns \(w_j\); if not, they remain at their outside option wage:
\[
  w_i^{\mathrm{out}} = 
    \begin{cases}
      w_i, & i\in\mathcal E,\\
      b,   & i\in\mathcal U\quad(\text{e.g.\ unemployment benefit }b<\overline w).
    \end{cases}
\]
Hence the \emph{expected utility} of applying to \(j\) is
\[
  \mathrm{EU}_{i,j}
    = p(\varphi)\,u_i(w_j)
    \;+\;\bigl[1 - p(\varphi)\bigr]\,u_i\bigl(w_i^{\mathrm{out}}\bigr).
\]

\subsection*{4. Application Rule with Risk}
Agent \(i\) now applies to every vacancy \(j\) for which
\[
  \mathrm{EU}_{i,j}
  \;\ge\; u_i\bigl(R_{i,t}\bigr),
\]
i.e.\ the expected gain in utility exceeds the utility of staying at the reservation wage.

Thus the number of applications is
\[
  A_{i,t}
    = \#\bigl\{\,j:\;p(\varphi)\,u_i(w_j)
      + [1 - p(\varphi)]\,u_i(w_i^{\mathrm{out}})
      \;\ge\; u_i(R_{i,t})\bigr\}.
\]

\subsection*{5. Discouragement Dynamics}
Define success in period \(t\) by
\[
  s_{i,t} = 
  \begin{cases}
    1, & \text{if $i$ obtains a job offer at $t$,}\\
    0, & \text{otherwise.}
  \end{cases}
\]
Discouragement \(D_{i,t}\)  evolves:
\[
  D_{i,t+1} \;=\; \rho_i\,D_{i,t} \;+\; (1-s_{i,t}), 
  \quad 0<\rho_i<1,
\]
with \(\rho_U<\rho_E\).

\color{red}

\subsection*{5. Effort Scaled Application Rule}
Define two effort factors:
\[
  e_{E,t}
  = \max\bigl\{0,\;\varphi\,\bigl(1 - \kappa_E\,D_{E,t}\bigr)\bigr\},
  \qquad
  e_{U,t}
  = \max\bigl\{0,\;1 - \kappa_U\,D_{U,t}\bigr\},
\]
with \(\kappa_U>\kappa_E\).  Note that  $\kappa_i\ $: sensitivity of application effort to discouragement. Therefore
- Employed: if \(D_{E,t}=0\), effort \(=\varphi\) (more tightness = more search).  
- Unemployed: effort only falls with discouragement.

Let  
\(\displaystyle \mathcal A_{i,t}^* = \{\,j: \mathrm{EU}_{i,j}\ge u_i(R_{i})\}\)  
be the set of ?worthwhile? vacancies.  Then the actual number of applications is
\[
  A_{i,t}
  = e_{i,t}\,\bigl|\mathcal A_{i,t}^*\bigr|
  \;=\;
  e_{i,t}\;\#\Bigl\{\,j:\;p(\varphi)\,u_i(w_j)
    +[1-p(\varphi)]\,u_i(w_i^{\mathrm{out}})
    \ge u_i(R_i)\Bigr\}.
\]
\color{red}

\end{document}
