\documentclass[hidelinks]{article}
\usepackage[a4paper, margin = 1in]{geometry}
\usepackage[utf8]{inputenc}
\usepackage{blindtext}
\usepackage{authblk}
\usepackage{xcolor}
\usepackage{hyperref}
\usepackage{doi}
\usepackage[skip=2pt]{caption}
\usepackage[style = authoryear]{biblatex} 
\usepackage{graphicx}
\usepackage{placeins}
\usepackage{import}
\usepackage{tabularx}
\usepackage{xcolor} % optional if you want background colors

\addbibresource{em_si_mdrc.bib}

\title{Literature Inventory: Job Search Behaviour}
\author{}

\begin{document}
\setlength\parskip{1em plus 0.1em minus 0.2em}
\maketitle

\begin{table}[h]
\centering
\renewcommand{\arraystretch}{1.2}
\begin{tabularx}{\textwidth}{|>{\raggedright\arraybackslash}X|>{\raggedright\arraybackslash}X|}
\hline
\multicolumn{2}{|c|}{\textbf{\Large Overview: Behaviour and the Job Search Process}} \\ \hline

\textbf{Personal or Contextual Factors} &
\textbf{Psychological Traits} \\
\hline
Age & Optimism (about labour market) \\
Gender & Locus of control \\
Family composition & Self-efficacy \\
Income \& Assets & Extraversion \\
Skills & Openness to Experience \\
Rural–urban & Rejection aversion \\
Employment status & Risk aversion \\
Tightness of local labour market & Impatience \\
Social pressure & (Mis)perception of ability \\
Network & Bounded or imperfect rationality \\
 & Self-regulation (re: negative emotions) \\
 & Self-motivation \\
\hline

\multicolumn{2}{|c|}{\textbf{Preferences}} \\ \hline
\multicolumn{2}{|l|}{Wage} \\
\multicolumn{2}{|l|}{Skills} \\
\multicolumn{2}{|l|}{Location} \\
\multicolumn{2}{|l|}{Industry/Sector} \\
\multicolumn{2}{|l|}{Identity alignment} \\
\multicolumn{2}{|l|}{Non-wage amenities} \\
\multicolumn{2}{|l|}{\emph{*Employed job seekers are more ambitious than unemployed}} \\
\multicolumn{2}{|l|}{\emph{*Unemployed maximize job offer chances while employed try to climb the job ladder}} \\
\hline

\multicolumn{2}{|c|}{\textbf{Behaviours}} \\ \hline
\multicolumn{2}{|l|}{Choice of search channel (online, newspaper, network, agency)} \\
\multicolumn{2}{|l|}{Delay search} \\
\multicolumn{2}{|l|}{Reservation wage} \\
\multicolumn{2}{|l|}{Other non-wage reservation preferences} \\
\multicolumn{2}{|l|}{Job search effort (hours searching, applications sent)} \\
\multicolumn{2}{|l|}{Job search intensity (planning, strategy)} \\
\multicolumn{2}{|l|}{``Aiming high or low''} \\
\multicolumn{2}{|l|}{Apply only for newer job postings} \\
\multicolumn{2}{|l|}{Concave search effort} \\
\multicolumn{2}{|l|}{Dynamism / Oscillating search effort} \\
\multicolumn{2}{|l|}{Cyclicality of job search in tight labour markets (in ``additional considerations'')} \\
\hline

\end{tabularx}
\end{table}

\tableofcontents

\section{Personal and Contextual Factors}

\subsection{Age} 

Older undemployed job seekers use less search channels than younger seekers in Belgian study by \parencite{guillemynAgeRelatedDifferencesJob2023}. They find that older (55+) Belgian job-seekers rely more on public employment agencies than networks relative to younger job-seekers. 

 

Much of the literature on age focuses on age discrimination in the labour market \parencite{richardsonAgeDiscriminationEvaluation2013, batinovicAgeismHiringSystematic2023, lippensStateHiringDiscrimination2023, watermannWithdrawingJobSearch2023}.

\subsection{Gender}

In Germany, when controlling for reservation wages, the gender wage gap essentially disappears. Suggests that reservation wages dictate the wage gap. Also suggest that the Big Five personality traits might influence these different reservation wages \parencite{caliendoGenderWageGap2017}.

 

Using Danish data, women and men who are observationally similar apply for different jobs which explains more than 70\% of the gender wage gap \parencite{fluchtmannGenderApplicationGap2021}.

 

Female graduate students tend to set lower reservation wages and accept jobs earlier than their male counterparts. However, they expressed greater job satisfaction upon employment than male counterparts because they did not aim “too high” when setting their reservation wages and other criteria. Men tended to wait longer to accept a job offer but also overestimated their employment success and on average accepted offers below their reservation wage more often than women \parencite{cortesGenderDifferencesJob2023}.

 

In Sweden, women set a smaller geographical search radius when applying for jobs than men did. This restriction led to fewer firm contacts and lower wage distribution of the vacancies available to them \parencite{erikssonLaborMarketConsequences2012}.

Analysing the Danish physician labour market, \parencite{fadlonCausalEffectsEarly2022} find that early career choices have long-lasting impacts on various life cycle outcomes including occupational choice, marriage market choices, and fertility. The persistent effects almost exclusively affect women whereas any effects for men are temporary.

Additional resources to be summarised here: \parencite{vanhooftPredictorsOutcomesJob2005, azmatGenderLaborMarket2014, fadlonCausalEffectsEarly2020, mcgeeGenderDifferencesReservation2023, kiesslingGenderDifferencesWage2024}. 

\parencite{delfinoBreakingGenderBarriers2024} vary recruitment messages containing photographs of male versus female employees and information about the share of workers receiving high evaluations in the past. They find that the photographs have little impact on men or women. Additionally, they find that men are "attracted" to the jobs that indicate a lower share of "talent recognition" which the authors claim speaks to "men's uncertainty about talent recognition in talent-dominated fields." In other words, the authors argue that "talent" is less likely to be recognised or appreciated in roles in which majority of workers receive a positive evaluation. \textcolor{red}{I must be honest that I do not find the explanations for the results satisfactory in this study...they seem contrived to support the results.}

Also to cite Alexia Delfino’s work on pink collar jobs and the important of perceived gender differences by occupation \href{https://www.lse.ac.uk/economics/Assets/Documents/job-market-candidates-2019-2020/BreakingGenderBarriers-Delfino.pdf}{(Source)}.

\parencite{diasCareersTimeUse2021} find no evidence that women's declining participation in paid work following childbirth is related to a "maximising joint income" hypothesis. Essentially, women are most likely to stop working after parenthood regardless of income relative to their spouse.


\parencite{olivettiChapter8Evolution2024} have a review chapter in the Handbook of Labor Economics on different contributors to the gender wage gap. This could be a key reference in our recognition that the emergent gender wage gap in our model is only a consequence of occupational choice and risk aversion. 

\parencite{niederleGenderCompetition2011} survey various laboratory and field studies to find that gender differences in competitiveness result from (1) differences in overconfidence  and (2) differences in competitiveness between men and women.  They claim that risk aversion has less to do with this difference. The observed effect is that women enter and win less competitions across the studies.
 

\subsection{Family composition}

Using Chilean job posting data, married males apply more intensely than their single counterparts; opposite is true for women \parencite{banfiDeconstructingJobSearch2019}.
 

In the US, reemployment of a male spouse is valued higher than a female spouse, facilitating the reemployment of a male spouse by directing time, space, and money to their job-search processes \parencite{raoIdealJobSeekerNorm2021}. 

 

\subsection{Income \& Assets}

\textcolor{violet}{There was less evidence on the income precarity effects on job search than I had expected...Only a few older sources. I might be using the wrong search terms?}

Home ownership restricts geographical radius of job search efforts of unemployed workers. Home owners do not seem to compensate for this restriction with additional search effort or behaviour \parencite{caliendoHomeownershipUnemployedJob2015}.

Financial hardship and shame: Unemployed individuals that face financial hardship feel significant shame and erosion of self-confidence (Sweden) \parencite{rantakeisuFinancialHardshipShame1999}. 
 

Assistance or unemployment insurance will affect job search intensity. \parencite{kruegerJobSearchUnemployment2010} find that job search effort depends on conditions of termination (ie. is there a likelihood a worker will be recalled, do they receive financial compensation, etc.). Furthermore, job search intensity/effort increases as any available UI benefits reach exhaustion. 

 

\parencite{dellavignaReferenceDependentJobSearch2017} propose non-constant search effort for unemployed individuals as they experience acute loss aversion relative to recent income immediately following a job loss but eventually grow accustomed to a lower income/being unemployed and the role of loss aversion then wears off slightly. They find a similar pattern of behaviour at the point of exhaustion of UI benefits. 

 

\parencite{lentzJobSearchSavings2005} study a “risk averse job seeker” and find that wealth and job search effort have a negative relationship. Job search increased with unemployment duration, a result that the author’s suggest depends heavily on the “risk aversion” of the agent modelled. 

 

In a study of the effectiveness of different labour market interventions in Finland found that the deterioration of one’s financial situation contributed to greater search activity \parencite{vuoriLabourMarketInterventions1999}.
  

\subsection{Skills}

Using PIAAC data, they investigate the effect of educational and skills mismatch on wages. Not immediately relevant for our work but touches on a broader literature of skills and educational mismatch \parencite{mateosromeroWageEffectsCognitive2017}.

Though not explicitly skill related, \parencite{ngaiGenderGapsRise2017} attribute a share of observed trends in women's hours and relative wages in recent decades to women having a comparative advantage in producing services. \textcolor{violet}{I personally do not like this framing as a "comparative advantage", but worded differently, this source could be useful in our discussion of gendered occupational choice.}

\subsubsection{Skills and Task Overlap}
Though not immediately related to job search behaviour per se, the literature on task overlap and task similarity for job-finding success is relevant to our work. In the case of \parencite{dabedEqualisingEffectsAutomation2025}.


\subsection{Rural-urban} 

The geographical region, country, or broader culture within which the job search occurs is relevant for the experience of job search. Job seekers in rural settings may need to rely more on social capital than job seekers in urban settings \parencite{matthewsSocialCapitalLabour2009}.

 

\parencite{sahinMismatchUnemployment2014} investigate the mismatch between vacancies and job seekers across sectors to determine how much of unemployment. They find that mismatch across industries and 3-digit occupations accounts for 1/3 of the total observed increase in the US unemployment rate in 2006 – 2009. They report that geographical mismatch played no role in increasing the unemployment rate according to their model.  

\subsection{Employment status}

In Sweden, employers tend to offer more interviews to already employed applicants rather than unemployed applicants \parencite{erikssonCompetitionEmployedUnemployed2006}.

Similarly, \parencite{trzebiatowskiUnemployedNeednApply2020} find that the length of unemployment (if information is available to an employer) can lower the probability that an applicant gets an interview request. 

Using evidence from laboratory, field, survey, and labour market panel data, \parencite{cohnFrequentJobChanges2021} find consistent evidence that a history of frequent job changes leads to less "callbacks" from employers in job search processes. 

\subsection{Tightness of local labour market}

Individuals tend to exhibit greater post-employment satisfaction if searching during tough economic times. In better economic times, job seekers tend to ruminate about whether they could have gotten more \parencite{bianchiBrightSideBad2013}.

 

Tightness in markets for jobs for which an unemployed job seeker fully qualifies in terms of task competencies is predictive of their unemployment duration. Task overlaps across jobs is unimportant for worker mobility between job markets \parencite{goosMarketsJobsTheir2019}. This paper evidently has implications for decision to look at occupational choice rather than skills similarity.  

\subsection{Social pressure}

As another example of the role of culture in job search, social pressure to find work is a stronger predictor of job search intentions in collectivist cultures than in individualistic cultures \parencite{vanhooftPredictorsOutcomesJob2005}. 

\subsection{Network}

Large networks convey helpful information to job seekers, promote informal job searching, and push workers to set higher reservation wages \parencite{caliendoSocialNetworksJob2011}. 

 

Social network effects on job searches and employment distinguishes systematically between “weak” and “strong” ties. Most studies find positive effects of re-employment through the use/presence of social networks only when qualifying the “strength” of the ties in the network in some way (varies across the literature). Another line of distinction that many papers draw is in which stage of the job search process an individual is in (applications, interviews, etc) as well as the diversity of the network (ie. job seekers will only be exposed to jobs that define the people in their social network \parencite{gargBeNotBe2011, barbulescuStrengthManyKinds2015}. 

 

Study of recent Swiss graduates found that networking led to lower search costs and more relevant job offers but not necessarily higher wages \parencite{franzenSocialNetworksLabour2006}.

 

However, the research is not unanimously in agreement about the “benefits” of social networks as discussed in the review by \parencite{wanbergJobSeekingProcess2020}.

 

Study in the Netherlands found that social pressure to search for a job is lower for individuals with families than single people. Also found that there was no meaningful difference in susceptibility to social pressure between men and women \parencite{vanhooftPredictorsOutcomesJob2005}.

 

Social stigma of unemployment is strong. Job-seekers’ sensitivity/awareness of this stigma is heterogeneous. This study finds that individual’s who are very conscious of this stigma tend to increase their search effort but do not have better re-employment chances than those less sensitive \parencite{krugSocialStigmaUnemployment2019},
 

Review of networking as job search behaviour \parencite{forretNetworkingJobSearchBehavior2018}. 


\section{Psychological Traits}

\subsection{Optimism (about labour market)}

Incomplete information about the labour market impacts job search success and by extension job search duration. Model proposed by \parencite{gonzalezEquilibriumTheoryLearning2010} demonstrate that mismatch of expectations and labour market reality can lead to greater negative affect, lower reservation wages, worse re-employment outcomes (with respect to expectations), and cause job-seekers to become discouraged. 

 

\parencite{adams-prasslPerceivedReturnsJob2023, cortesGenderDifferencesJob2023} both either investigate or observe mismatched expectations about the labour market. 

 

\parencite{spinnewijnUnemployedOptimisticOptimal2015} Explores how the design of unemployment insurance schemes might be affected by the observed relationship between biased beliefs about employment prospects incentivises job-seekers to search too little, save too little for unemployment, and deplete their savings might. 

\subsection{Locus of control}

Using panel of unemployed individuals in Germany – individuals with internal locus of control search more and set higher reservation wages \parencite{caliendoLocusControlJob2015}.

 

Internal locus of control is an important predictor of whether people in low-wage jobs can climb the wage ladder \parencite{schnitzleinLocusControlLowwage2016}.  

 

In the context of work-related training, GSOEP results show that higher internal locus of control is predictive of taking up general, but not specific, training – locus of control is posited to influence uptake of training via effect on workers’ expectations about future wage increases rather than actual wage increases \parencite{caliendoLocusControlInvestment2016}.

 

Individuals with an internal locus of control are predicted to search across larger geographic areas and migrate more frequently \parencite{caliendoLocusControlInternal2019}. 

 

\parencite{mcgeeSearchEffortLocus2016} report in a lab experiment that internal locus of control/self-efficacy beliefs increase search effort and reservation wages when the return to effort is uncertain.  

 

\subsection{Self-efficacy}

\parencite{liuSelfregulationJobSearch2014} compare job search self efficacy and employment self-efficacy. Job search progress (positive outcomes) tended to reinforce both forms of self-efficacy leading to diverging job search effort effects (employment efficicacy tended to decrease search effort whereas job search self-efficacy increased search effort).

Find that job search effort increases with openness to experience; agreeableness; positive attitudes toward search; self-efficacy; financial need; social pressure \parencite{vanhooftMovingJobSearch2013}.

 

Investigates perceived returns to job search. Systematically, job-seekers update their perceived returns to job search when they receive positive reinforcement and vice versa \parencite{adams-prasslPerceivedReturnsJob2023}. In the UK sample analysed, the authors find that respondents were on average over-optimistic about their job finding rates.  

 

\subsection{Extraversion}

\parencite{turbanEffectsConscientiousnessExtraversion2009} found that extraversion’s regulatory effect on emotions led to greater job search activities and success.  

\subsection{Openness to Experience} 

Systematic review of the big 5 personality traits and their relationship to earnings \parencite{alderottiBigFivePersonality2023}.


Some studies look at openness to experience and job performance trajectories finding some positive associations/correlations \parencite{minbashianOpennessExperiencePredictor2013a}. Further research looks at the effect of openness to experience (among Big 5 personality traits) on job instability finding that individuals with a greater “openness to experience” tend to engage in ‘job hopping’ \parencite{willeVocationalInterestsBig2010}.

\subsection{Rejection aversion}

\textcolor{violet}{This was an aspect that I had originally categorised as a manifestation of “risk aversion.” But now I wish to see whether there is literature specifically about rejection aversion…but have struggled so far to find central sources.  I’ve given it its own section here as a reminder to explore it further.}

\subsection{Risk aversion}

In the context of training investment, risk averse workers engage in training only if it insures them against future losses, whereas training as an “investment risk” with an uncertain outcome leads to the opposite behaviour. Risk averse individuals take up less general training due to the uncertainty of gains from training \parencite{caliendoRiskPreferencesTraining2023}.

 

Risk aversion is a useful predictor of whether an individual will already be in or will attempt to reach a high-earnings job \parencite{boninCrosssectionalEarningsRisk2007}. Although the central research question of \parencite{cortesGenderDifferencesJob2023} is not specifically focused on risk aversion, they do claim that female students tend to set lower reservation wages and accept job offers earlier potentially as a result of being on average more risk averse. 

 

Furthermore, risk aversion has similarly been determined to have strong associations with demographic and socioeconomic characteristics like gender, income, and cognitive ability \parencite{heckmanEffectsCognitiveNoncognitive2006, erikssonLaborMarketConsequences2012, cortesGenderDifferencesJob2023}.

 

\subsection{Impatience}

Workers who are more impatient (defined as setting a high discount rate on future experiences/consumption/earning) search less intensively and set lower reservation wages \parencite{dellavignaJobSearchImpatience2005}.

\subsection{(Mis)perception of ability}

Several studies either centrally explore or attribute research findings to misperceptions of one’s abilities or employment success \parencite{spinnewijnUnemployedOptimisticOptimal2015, muellerJobSeekersPerceptions2021, adams-prasslPerceivedReturnsJob2023, cortesGenderDifferencesJob2023}.


In a study using Chilean job ad data, the authors find that men tend to apply for jobs they are more unfit to do \parencite{banfiDeconstructingJobSearch2019}.

 \parencite{bandieraMenAreMars2022} find somewhat discordant evidence between this difference in confidence between men and women. Though an expert survey indicated beliefs that men are overconfident and women underconfident, their meta-analytic study revelas no such result. 

\subsection{Bounded or imperfect rationality}

\textcolor{violet}{...}
 

\subsection{Ability to deal with negative emotions/self-regulation}

\parencite{lopez-kidwellWhatMattersWhen2013, kreemersDealingNegativeJob2018} investigate the role of negative job search experiences on job search intensity/effort as well as the role of being able to “regulate” negative emotions and recover/continue searching. 

 
In a survey-based experiment over 24 weeks, \parencite{kruegerJobSearchEmotional2011} find that job search time declines over unemployment duration. Subjects surveyed expressed significant dissatisfaction and unhappiness with their lives and that searching for jobs exacerbated these negative emotions. 

\subsection{Self-determined motivation}

Find that self-determined motivation is key for ensuring both quality search strategy and high search effort \parencite{damottaveigaRoleSelfdeterminedMotivation2016}.

 

\section{Preferences} 

\subsection{Wage}

Find that job seekers respond to higher wages of otherwise identical vacancies \parencite{belotHowWageAnnouncements2022}.

Using online job board data, authors find that high-wage ads attract more applicants \parencite{banfiDeconstructingJobSearch2019}.

\parencite{haltiwangerCyclicalJobLadders2018} find evidence of a firm wage ladder (workers transition from lower- to higher-paying firms). However, they found that during the Great Recession, the “firm wage ladder collapsed with net worker reallocation to higher wage firms falling to zero.” 

Using, data from a jobs ad portal in India, \parencite{chaturvediGenderedLanguageJob2025} find that advertised wages are lower among employers who prefer women, women make up a greater share of applicants to lower-wage job ads, and employer preferences (explicit gender requests) explain 7\% of the gender wage gap in applied-for jobs. Analysing "gendered" language in the job ads, the authors find that hard-skills related female-gendered words attract more female applications but offer lower wages whereas male-gendered words indicating low flexibility in work schedules/arrangements had higher wage returns but lower female applicant shares. 

\subsection{Skills}


Several studies confirm the role of skills matches both in determining job search strategies but also in hiring success \parencite{schoenbergHowGeneralSpecific2006, dawsonSkilldrivenRecommendationsJob2021, nayaDesigningLaborMarket2021, neffkeSkillMismatchCosts2022, biedFairnessJobRecommendations2023, biedJobRecommendationAll2023, soaresmartinsnetoForcedDisplacementOccupational2023}.
 

\subsection{Location}

Finds that relocation assistance improves geographical mobility and broadens search radius of workers. Has adverse local effects as well. Proposes that spatial search frictions drive unintended adverse labour market consequences. Furthermore, subsidy recipients tend to more to regions with better economic conditions and increase wage levels and employment stability \parencite{caliendoReturnLaborMarket2017}.

\parencite{marinescuMismatchUnemploymentGeography2018} use data from the US CareerBuilder.com and confirm that job-seekers have a distaste for more distant jobs. However, the authors argue that this only has a minor effect on aggregate unemployment (would only decrease unemployment by 5.3\%). 

 

Using Swedish CV database, the authors find that women are more restrictive than men in their choice of search area and are more averse to applying for jobs in metropolitan areas. This affects the amount of contacts they get from firms as well as the wage distribution of offers or available vacancies \parencite{erikssonLaborMarketConsequences2012}.

Using French labour market data, the authors observe that unemployed women both set lower reservation wages and a shorter acceptable maximum commute time. They then model indifference curves between the two preferences and find that women value commute around 20\% more than men. They find that once re-employed women are paid 4\% less and commute 12\% less than men. Using job application data they also confirm that women don’t face less demand from far-away employers but they themselves set the restriction \parencite{lebarbanchonGenderDifferencesJob2021}.

\parencite{limLocationMajorBarrier2023} demonstrate that, in the context of the green transition, fossil fuel workers are not currently located in areas that offer green employment posing a major barrier to a Just Transition. Furthermore, reports that “the percentage of Americans who moved across state lines each year fell by 50\% during the period 1990-2018 \parencite{rickardEconomicGeographyPolitics2020}. 

 
 \parencite{hendricksonCounteringGeographyDiscontent2018} outlines the emergence of two US geographies: large, diverse, thriving metropolitan regions vs. “homogenous small towns and rural areas struggling under the weight of economic stagnation and social decline.” The authors additionally discuss the “identities” that this stratification builds in the US (ie. geographies of discontent and resentment of “coastal city elites.” This is likely relevant to our interest in how identities can serve as a barrier to transitioning. 

 

\subsection{Industry/Sector}

\subsection{Identity alignment}

\subsection{Non-wage amenities}

Non-wage amenities seem to matter considerably in the diverging preferences of men and women. Using Danish data, \parencite{adamsContributionEmployeeLedEmployerLed2025} find that women are employed in higher numbers in employee-led flexibility occupations (where employees are free to set their work schedule rather than employers being flexible to last-minute changes to the work schedule). Women experience a higher return to working in such occupations than men. 

\parencite{adams-prasslFirmConcentrationJob2023} analyse job vacancy text using a supervised machine learning approach to assess the value of schedule flexibility. They find the such a non-wage amenity is valuable only when offered alongside a stable contract such that workers can avoid earnings variation. Essentially, non-wage amenities matter only as an additional offer above a stable wage contract. 

\section{Behaviours}

Key overview of the basics of job search behaviour \parencite{vandenbergEconomicJobSearch2018} defined as consisting of reservation wage, imperfect information about jobs, search effort, income, and chosen search channels. 

\subsection{Choice of search channel (online, newspaper, network, employment agency)}

Unemployed job seekers can engage in job searches in different ways: formal vs. informal; preparatory or active; focused/exploratory/haphazard \parencite{songJobSearchBehaviorUnemployed2018}.

Many interesting papers have explored the effect of biased job recommendation resources (whether online or in person) on labour market outcomes \parencite{rusClosingGenderWage2022, biedFairnessJobRecommendations2023, biedJobRecommendationAll2023}.

\subsection{Delay}

Find that unemployed workers are often dealing with despair, burnout, depression which can delay job search \parencite{amundsonDynamicsUnemploymentJob1982}.

Individual mental health consequences can cause delay or negative affect toward job search \parencite{paulIndividualConsequencesJob2018}.

Job-seekers who become involuntarily unemployed are found to delay job search until they have “grieved” the loss of their job. Job-seekers who left willingly tend to exhibit delay because they take a sabbatical leave or intended to take a break \parencite{songJobSearchBehaviorUnemployed2018}.

In a study on the effects of unemployment benefit duration and job search delay, the authors provide back-of-the-envelope evidence that there are substantial returns to early search effort (ie. search effort is higher earlier in unemployment duration) \parencite{lichterBenefitDurationJob2021}.

\subsection{Reservation wage} 

Using longitudinal panel data from unemployed workers in New Jersey, reservation wages start out too high and then decline too slowly suggesting that job-seekers persistently misjudge their prospects or anchor their reservation wage on previous wage \parencite{kruegerContributionEmpiricsReservation2016}. A study of French administrative data replicate the result that reservation wage does not decrease very quickly in response to changes in duration of unemployment benefits \parencite{lebarbanchonUnemploymentInsuranceReservation2019}. They argue that  this is likely because workers have reference-dependent wages as argued in \parencite{koenigReservationWagesWage2016, dellavignaReferenceDependentJobSearch2017}. \parencite{koenigReservationWagesWage2016} argue in a CEPR working paper that reference-dependent wages might contribute to the fact that wages exhibit little cyclicality. 

 Using German panel data, \parencite{bonaccolto-topferGenderDifferencesReservation2024} find that women set lower reservation wages than men translating into substantial gender pay gaps. These gaps are larger at the top of the reservation wage distribution and between individuals with children and high educational attainment. Apart from the parenting challenge, these insights are quite interesting and hint at potential discrimination or other ofrces at 'higher' levels of employment that might be linked to greater flexibility in wage premiums in higher-skilled roles (\textcolor{red}{My hypothesis so not sure...}.

Experimental and econometric evidence finds that the reservation wages of unemployed job-seekers decline in search duration \parencite{brownRealTimeSearchLaboratory2011}.

Reservation wages decline with search effort. Authors posit that search duration allows job-seekers to learn about the true wage distribution of relevant vacancies \parencite{burdettDecliningReservationWages1988}.

\parencite{kudlyakSystematicJobSearch2013} use high-frequency panel data on job applications in the US to demonstrate that search is systematic. Education predicts sorting across vacancies at the beginning of an unemployment spell but after several weeks of unemployment job seekers lower their reservation wage/wages of their chosen applications. It is found that later in the unemployment duration, job-seekers apply for jobs that were the first-week application choices of less educated workers.  

 

\subsection{Other non-wage preferences}

 

\subsection{Job search effort (hours spent searching, applications sent)}

Notably, across works surveyed for this literature review “search effort” is rarely centrally studied but one of the most common indicators of choice when examining the effect of any of the mechanisms/behavioural markers/identity markers mentioned on job searches and employment \parencite{dellavignaJobSearchImpatience2005, wanbergJobSearchGrind2010, lopez-kidwellWhatMattersWhen2013, vanhooftMovingJobSearch2013, mcgeeSearchEffortLocus2016, songJobSearchBehaviorUnemployed2018, vandenbergEconomicJobSearch2018, wanbergJobSeekingProcess2020}. For example, (already cited in other sections, but relevant to this block), the findings of \parencite{adams-prasslPerceivedReturnsJob2023} rely on a concept of “job search effort” measured as hours spent searching. 

Using data from Swiss public employment offices, \parencite{zuchuatDurationDependenceFinding2023} find that job applications and job interviews decrease with time but job offers following an interview increase with duration. "A model with statistical discrimination by firms and learning from search outcomes by workers replicates these empirical duration patterns closely. The structurally estimated model predicts that 55 percent of the decline in the job finding rate is due to "true" duration dependence, while the remaining 45 percent is due to dynamic selection of the unemployment pool. "

\subsection{Job search intensity (planning, strategy)}

The authors of \parencite{vanhooftMovingJobSearch2013} note the singular focus on job search effort/intensity but propose that “intensity” defined by its “quality” is critical as well. 

Information provision had marginal effects on job seekers’ employment prospects in a German field study. Suggests that bounded rationality likely plays a role in job searches as job-seekers do not optimise search even when information is provided on how to improve the search process \parencite{altmannLearningJobSearch2018}.

Unemployed individuals pursuing reemployment achieve better job quality upon re-employment when engaging in goal establishment and planning; preparation (interviews, resumes, etc.); are able to regulate negative emotions or affect to job search process; and learn or improve their strategies or expectations with experience \parencite{vanhooftHowOptimizeJob2022}.

 

\subsection{“Aiming high or low”}

In a study of Chilean job post data, employed seekers are more ambitious, they tend to apply to jobs requiring more skils and education than they have as well as jobs with higher wages \parencite{banfiDeconstructingJobSearch2019}.

 

\subsection{Apply only for newer job postings}

 

\subsection{Concave versus convex search effort} 

The empirical literature on this seems inconclusive. \textcolor{violet}{Given our analysis using the Displaced Workers Supplement, we might be able to contribute empirically to this question as well.}  

\parencite{kruegerJobSearchUnemployment2010} find a concave-shaped job search that peaked when UI approached exhaustion while those ineligible for UI showed a constant search over time.  
 

\parencite{kruegerJobSearchUnemployment2010, kruegerJobSearchEmotional2011} find that job search intensity decreases over time for unemployed individuals. Wanberg’s work is also cited in \parencite{wanbergJobSeekingProcess2020}.
 

In a study of Chilean job post data, search effort declines over time. Furthermore, unemployed workers tend to relax their requirements/preferences over time including wage \parencite{banfiDeconstructingJobSearch2019}. 
 
\parencite{fabermanIntensityJobSearch2019} find that within an individual search spell, search intensity declines continuously they also find that longer-duration job seekers search more intensely throughout their search. They tend to be older, male, nonemployed, and live in areas with weaker labor markets. Their findings contradict standard assumptions of labor search models . 


 

\subsection{Dynamism/Oscillating Search Effort}

Daily dynamics of job search. If unsuccessful one day, workers tend to overcompensate in effort the following day \parencite{wanbergJobSearchGrind2010}.

 
Review of the literature on job search temporality and persistence – job search as a dynamic process \parencite{songJobSearchBehaviorUnemployed2018}.

 
Job seekers increase search intensity as their UI benefits near exhaustion or graduation looms, in the case of unemployed workers and students, respectively \parencite{saksChangeJobSearch2000, cortesGenderDifferencesJob2023}.
 

\subsection{Additional Considerations}

\subsection{Cyclicality of the Labour Market and Behaviour}

Shows that search behaviour of the employed can crowd out the unemployed especially during periods of economic recovery when the gains to on-the-job search are perceived to be higher \parencite{eeckhoutUnemploymentCycles2019}. A similar insight using UK data reported by \parencite{branschCyclicalityOnthejobSearch2024} finds that on-the-job search is cyclical. They emphasize that this is more related to job-ladder-motivated searches than precautionary search motives. \textcolor{violet}{This potentially challenges our interpretation/motivation for OTJ cyclical search in the model as we imply that employed individuals are surveying the "state" of the labour market. This could be consistent if we do not apply a motivation to the 'surveying'}.
 
Using online data on search and placement intensity of employment agencies and firms \parencite{hutterCyclicalityLabourMarket2021}  finds that search and placement intensity are both pro-cyclical while job-seekers’ search intensity is counter-cyclical. They find that this relationship breaks down in the Covid era in where workers seem to exhibit pro-cyclical search behaviour. 

 \parencite{mukoyamaJobSearchBehavior2018} demonstrate that search effort tends to dampen rather than amplify labor market fluctuations due to the counter-cyclicality of search effort. 

\subsection{Employed versus Unemployed}

Unemployed individuals search more intensely, using more diverse sources, and collect more offers than employed seekers \parencite{songJobSearchBehaviorUnemployed2018}.


On-the-job search is pervasive and more intense at lower rungs of the job ladder; the employed are at least three times more effective than the unemployed in the job search process; the employed receive better job offers than the unemployed \parencite{fabermanJobSearchBehavior2022}. In a general equilibrium model, they also demonstrate the feedback effects of endogenous and elastic job search effort to macro-level fluctuations ie. that there is a feedback effect between search effort, vacancy postings, and thus macro-level responses like productivity. 

 

The attitude-intention-behavior relationship is stronger for unemployed than employed seekers \parencite{hooftPredictorsJobSearch2004}.

 
Employers tend to be very averse to hiring job-seekers that have been unemployed for longer periods of time, even when they might have more compatible skills or industry experience than other applicants \parencite{ghayadEscapingUnemploymentTrap2013}.

\subsection{Presence of Unemployment Insurance, Benefits, or Active Labour Market Policies} 

In Germany, unemployed workers can engage in marginal employment to supplement their income. Marginal employment is found to have higher job-finding probability and higher post-unemployment job outcomes (stability and wage). However, it does seem to prolong the duration \parencite{caliendoEarningsExemptionsUnemployed2016} of unemployment spells. 

 

There seems to be a considerable amount of studies about “spikes” in search effort at the point of UI benefit exhaustion \parencite{kruegerJobSearchUnemployment2010, dellavignaEvidenceJobSearch2020, dellavignaReferenceDependentJobSearch2017, marinescuUnemploymentInsuranceJob2021}. \parencite{cardSpikeBenefitExhaustion2007} suggest that this literature suffers from measurement error bias (ie. results are very sensitive to indicator choice and definition).  


Several studies find that UI benefit extensions following the Great Recession had a significantly positive effect on unemployment duration and kept long-term unemployed workers in the labour force that would have otherwise exited \parencite{kroftLongTermUnemploymentGreat2016, farberExtendedUnemploymentBenefits2013, rothsteinUnemploymentInsuranceJob2011}.

Additional resources (Mortensen, 1977)

\textcolor{violet}{Important new review paper of the empirical research \parencite{lebarbanchonJobSearchUnemployment2024}.}

\parencite{gollerActiveLaborMarket2025} finds that active labor market (\textit{job-training, reducing impediments,} and (\textit{private) placement services}) policies can help individuals who experience long-term unemployment acheive re-employment. 


\subsection{Review Search and Matching Models Literature}

Resources to consider in relation to testing the validity of search-and-matching models \parencite{denhaanTurbulenceUnemploymentJob2005, sundeEmpiricalMatchingFunctions2007, hutterMismatchForecastingPerformance2017}.

\section{Important meta-analytic studies and literature reviews}

Important meta-analytic studies and literature reviews: 

\textbf{Job-Search Behavior of the Unemployed: A Dynamic Perspective \parencite{songJobSearchBehaviorUnemployed2018}}

This work both outlines key research up until 2018 of job search behaviour and how it is conceptualised across various disciplines. Notably, the authors also outline five central model alternatives for job search: sequential (job-seekers follow a sequential process based on gathering information – ex. identifying an ideal occupation, planning the search, searching for and selecting the job, and confirming and committing to the decision), learning (The learning model suggests that job seekers’ increased learning over time may change job-search behaviors and strategies), emotional (The emotional model explains changes in job-search activities based on emotional reactions. This differs from cognitively focused models in its special attention to emotions and stresses. Job search is often associated with anxiety, frustration, setback, rejection, and future uncertainty), phase model of self-categorization and coping (The phase model of self-categorization and coping explains how unemployed individuals self-categorize and recategorize their social identity during unemployment phases and how categorizations affect coping strategies. Identity in many observations of this model is defined as an occupational identity that individuals find it difficult to let go of), and self-regulatory models (Recent job-search literature tend to conceptualize job search as a dynamic, goal-directed, self-regulatory process – this theory employs personality, motivational, and emotional predictors of job search strategy) of job search.  

\textbf{Job Search and Employment Success: A Quantitative Review and Future Research Agenda \parencite{vanhooftJobSearchEmployment2021}.} Builds on earlier work by \parencite{kanferJobSearchEmployment2001}. (See Figures \ref{fig:vanhooft_fig1} and \ref{fig:vanhooft_tbl7} below of concepts reviewed in the paper): 


\begin{figure}[ht]
    \centering
    \caption{Overview of job search concepts included in the meta-analysis of van Hooft et al. 2021 
(Manuscript Figure 1)}
    \includegraphics[scale = 0.6]{lit_review_elements/van_hooft_2021_fig1.png}
    \label{fig:vanhooft_fig1}
\end{figure}

\FloatBarrier


\begin{figure}[ht]
    \centering
    \caption{Overview of relationship with antecedent variables on job search - meta-analysis of van Hooft et al. 2021 
(Manuscript Table 4)}
    \includegraphics[scale = 0.6]{lit_review_elements/van_hooft_2021_tbl7.png}
    \label{fig:vanhooft_tbl7}
\end{figure}

\FloatBarrier
 

\textbf{Job Seeking: The Process and Experience of Looking for a Job \parencite{wanbergJobSeekingProcess2020}}

\textbf{Review of networking as job search behaviour \parencite{forretNetworkingJobSearchBehavior2018}}

\textbf{Job-Search Behavior as a Multidimensional Construct: A Review of Different Job-Search Behaviors and Sources \parencite{hoyeJobSearchBehaviorMultidimensional2018}}

\textbf{Building a network theory of social capital \parencite{linBuildingNetworkTheory2001}} 

An early review of search theories and models as alternatives to supply and demand models allowing for the modelling of non-zero unemployment \parencite{rogersonSearchTheoreticModelsLabor2005}. They also outline the “margins” of search behaviour along which labour market policies can influence job search behaviour: reservation wage choice, search intensity, entry, etc. 

\section{Leading Authors}

An inventory of names that come up most often in the search. \textcolor{red}{This might be incomplete and the easiest way for Stefi to complement.}

\begin{itemize}
    \item Connie Wanberg 
    \item Marco Caliendo 
    \item Abdifatah Ali 
    \item Zhaoli Song 
    \item Stefano DellaVigna 
    \item Eriksson \& Lagerström (Swedish context) 
    \item Jason Faberman 
    \item Marianna Kudlyak 
    \item Alan Krueger 
    \item Andreas I. Mueller
    \item Thomas Le Barbanchon 
    \item Bruno Crepon 
    \item Edwin Van Hooft 
    \item Barbara Petrongolo
\end{itemize}


\section{General arguments for the benefit of considering behaviour/psychology}

\textcolor{violet}{I think there are some by the Complexity group as well particularly related to ABMs.}

\parencite{mcfadyenEconomicPsychologicalModels1997, charnessChapterLabLabor2011, dohmenBehavioralLaborEconomics2014, caliendoUnobservableUnimportantRelevance2017}


Also very interested in this paper but could not find a published version: https://web.econ.ku.dk/nharmon/docs/glenny2021dynamics.pdf 

\printbibliography

\end{document}