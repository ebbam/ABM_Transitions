\setcounter{table}{0}
\setcounter{figure}{0}
%\renewcommand{\lastpage}{\lastpageref{pagesLTS.roman}}
%\pagenumbering{roman}
%\setcounter{page}{1}
\renewcommand{\thetable}{S\arabic{table}}
\renewcommand{\thefigure}{S\arabic{figure}}
%\renewcommand{\thepage}{S\arabic{page}}

\date{}


\begin{appendices}
\section*{Supplementary Materials}

\subsection*{Potential Methodological Challenges}
\label{si_section:Challenges2}
Three key challenges emerge when aiming to incorporate the proposed micro foundations into an agent-based labour market model: intractability, interactive effects, and validation.

\subsubsection*{\emph{Intractability \& Interactive Effects}}
By construction, agent-based models are tools used to represent economies and societies as complex systems without the assumption of perfectly predictable relationships between causes and effects. However, for the purpose of this study, I am still interested in understanding \emph{where} eventual outputs come from. In particular, this study, by incorporating elements into an already well-calibrated and -validated labour market model, the overall and relative utility of each of the components introduced will need to be justified. Therefore, understanding how the output of our model depends on these additional micro foundations will be critical to making an argument for their inclusion in future work.\\

Additionally, the micro foundations incorporated in this work do not operate independent of each other. For example, both risk aversion and impatience could be viewed as preferences. Additionally, expressed preferences for wage and location are heavily influenced by risk aversion and impatience. One might set a reservation wage or commute time more stringently (loosely) as a result of high risk aversion (high impatience). Unpacking these interactive and dependent effects carefully in the model construction and analysis of output will be of paramount importance. These potentially irreducible interactive effects might indeed dictate potential limitations of the work.

\subsubsection*{\emph{Validation}}
Finally, the current applications of the original occupational mobility network are validated against the stylised fact of the Beveridge curve. This means that the model already has the ability to provide important insight into aggregate labour market changes resulting from changes to labour demand without the incorporation of the proposed micro-foundations. This speaks highly of the robustness of the original model and its ability to understand \emph{how} an economy adjusts to changes in absolute and relative labour demand via aggregate or occupation-level unemployment levels and growth rates. However, this analysis does not allow a researcher to ascertain \emph{why} the economy adjusts the way it does from the perspective of human behaviour. For example, are slow adjustment times consequences of high risk aversion, stringent worker preferences in job searches, location? This work will need to incorporate alternative or additional validation methods to justify this work's added value in the literature.\\
% \color{red} Note from Stefi: Again, here I should consider the discussion on mechanisms \-- i.e. mechanisms are the key elements differentiating our model and therefore I need to make sure that I can make out of sample predictions of labour mkt dynamics on the basis of data on the mechanisms (or at least this is how I see it). \color{black}

Thus far, I have identified two possible validation alternatives. First, del Rio-Chanona et al's model is unable to adequately reproduce the real-world behaviour of long-term unemployment in that certain individuals or groups get caught in long-term unemployment despite the availability of suitable vacancies. Incorporating the outlined behaviours which inherently add friction to the labour market adjustment process and constitute ``irrational'' behaviour of agents, might reproduce this real-world long-term unemployment rate effect. Second, the gender wage gap is a stylized fact of virtually every labour market in the world. Varying risk aversion by gender might lend itself to validation against the gender wage gap. 

\end{appendices}